% DO NOT EDIT - automatically generated from metadata.yaml

\def \codeURL{https://github.com/Scofield626/DDC-P4-BMv2}
\def \codeDOI{}
\def \dataURL{}
\def \dataDOI{}
\def \editorNAME{}
\def \editorORCID{}
\def \reviewerINAME{}
\def \reviewerIORCID{}
\def \reviewerIINAME{}
\def \reviewerIIORCID{}
\def \dateRECEIVED{}
\def \dateACCEPTED{}
\def \datePUBLISHED{}
\def \articleTITLE{[Re] Ensuring Connectivity via Data Plane Mechanisms}
\def \articleTYPE{Replication/Computer Networks}
\def \articleDOMAIN{}
\def \articleBIBLIOGRAPHY{bibliography.bib}
\def \articleYEAR{2021}
\def \reviewURL{}
\def \articleABSTRACT{\textbf{Abstract} Three years have passed since ReScience published its first article and since September 2015, things have been going steadily. We're still alive, independent and without a budget. In the meantime, we have published around 20 articles (mostly in computational neuroscience \& computational ecology) and the initial has grown from around 10 to roughly 100 members (editors and reviewers), we have advertised ReScience at several conferences worldwide, gave some interviews and we published an article introducing ReScience in PeerJ. Based on our experience at managing the journal during these three years, we think the time is ripe for proposing some changes.}
\def \replicationCITE{}
\def \replicationBIB{}
\def \replicationURL{}
\def \replicationDOI{}
\def \contactNAME{Zhengqing Liu}
\def \contactEMAIL{zhengqing.liu@polytechnique.edu}
\def \articleKEYWORDS{rescience c, rescience x}
\def \journalNAME{ReScience C}
\def \journalVOLUME{4}
\def \journalISSUE{1}
\def \articleNUMBER{}
\def \articleDOI{}
\def \authorsFULL{Zhengqing Liu, Romain Jacob, Roland Schmid and Laurent Vanbever}
\def \authorsABBRV{Z. Liu, R. Jacob, R. Schmid and L. Vanbever}
\def \authorsSHORT{Liu, Jacob, Schmid, Vanbever}
\title{\articleTITLE}
\date{}
\author[1,2]{Zhengqing Liu}
\author[1]{Romain Jacob}
\author[1]{Roland Schmid}
\author[1]{Laurent Vanbever}
\affil[1]{ETH Zurich, Switzerland}
\affil[2]{École Polytechnique, Paris, France}
